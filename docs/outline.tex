%  LaTeX support: latex@mdpi.com 
%  For support, please attach all files needed for compiling as well as the log file, and specify your operating system, LaTeX version, and LaTeX editor.

%=================================================================
\documentclass[journal,article,submit,pdftex,moreauthors]{Definitions/mdpi} 
\usepackage[xcolor=pst]{pstricks}
% MDPI internal commands - do not modify
\firstpage{1} 
\makeatletter 
\setcounter{page}{\@firstpage} 
\makeatother
\pubvolume{1}
\issuenum{1}
\articlenumber{0}
\pubyear{2023}
\copyrightyear{2023}
%\externaleditor{Academic Editor: Firstname Lastname}
\datereceived{ } 
\daterevised{ } % Comment out if no revised date
\dateaccepted{ } 
\datepublished{ } 
%\datecorrected{} % For corrected papers: "Corrected: XXX" date in the original paper.
%\dateretracted{} % For corrected papers: "Retracted: XXX" date in the original paper.
\hreflink{https://doi.org/} % If needed use \linebreak
%\doinum{}
%\pdfoutput=1 % Uncommented for upload to arXiv.org

%=================================================================
% Add packages and commands here. The following packages are loaded in our class file: fontenc, inputenc, calc, indentfirst, fancyhdr, graphicx, epstopdf, lastpage, ifthen, float, amsmath, amssymb, lineno, setspace, enumitem, mathpazo, booktabs, titlesec, etoolbox, tabto, xcolor, colortbl, soul, multirow, microtype, tikz, totcount, changepage, attrib, upgreek, array, tabularx, pbox, ragged2e, tocloft, marginnote, marginfix, enotez, amsthm, natbib, hyperref, cleveref, scrextend, url, geometry, newfloat, caption, draftwatermark, seqsplit
% cleveref: load \crefname definitions after \begin{document}

%=================================================================
% Please use the following mathematics environments: Theorem, Lemma, Corollary, Proposition, Characterization, Property, Problem, Example, ExamplesandDefinitions, Hypothesis, Remark, Definition, Notation, Assumption
%% For proofs, please use the proof environment (the amsthm package is loaded by the MDPI class).

%=================================================================
% Full title of the paper (Capitalized)
\Title{Modeling Complex Interactions in Tree-Mycorrhizal Fungus Networks: Community Analysis and Centrality Measures}

% MDPI internal command: Title for citation in the left column
\TitleCitation{Title}

% Author Orchid ID: enter ID or remove command
\newcommand{\orcidauthorA}{0000-0000-0000-000X} % Add \orcidA{} behind the author's name
%\newcommand{\orcidauthorB}{0000-0000-0000-000X} % Add \orcidB{} behind the author's name

% Authors, for the paper (add full first names)
\Author{Firstname Lastname $^{1,\dagger,\ddagger}$\orcidA{}, Firstname Lastname $^{2,\ddagger}$ and Firstname Lastname $^{2,}$*}

%\longauthorlist{yes}

% MDPI internal command: Authors, for metadata in PDF
\AuthorNames{Firstname Lastname, Firstname Lastname and Firstname Lastname}

% MDPI internal command: Authors, for citation in the left column
\AuthorCitation{Lastname, F.; Lastname, F.; Lastname, F.}
% If this is a Chicago style journal: Lastname, Firstname, Firstname Lastname, and Firstname Lastname.

% Affiliations / Addresses (Add [1] after \address if there is only one affiliation.)
\address{%
$^{1}$ \quad Affiliation 1; e-mail@e-mail.com\\
$^{2}$ \quad Affiliation 2; e-mail@e-mail.com}

% Contact information of the corresponding author
\corres{Correspondence: e-mail@e-mail.com; Tel.: (optional; include country code; if there are multiple corresponding authors, add author initials) +xx-xxxx-xxx-xxxx (F.L.)}

% Current address and/or shared authorship
\firstnote{Current address: Affiliation 3.} 
\secondnote{These authors contributed equally to this work.}
% The commands \thirdnote{} till \eighthnote{} are available for further notes

%\simplesumm{} % Simple summary

%\conference{} % An extended version of a conference paper

% Abstract (Do not insert blank lines, i.e. \\) 
\abstract{Mycorrhizal networks are important underground networks that form between plant roots and mycorrhizal fungi. This structure plays a vital role in nutrient cycling, plant communication and growth, especially in nutrient-poor environments. Graph theory can be used to model and analyze these networks, with plants represented as nodes and their connections through mycorrhizal fungi represented as edges. Scientists can use various graph theoretical parameters, such as degree distribution, clustering coefficient and path length, to gain a better understanding of the network's functioning and the interactions between trees and their associated fungi.
In this study, we aim to investigate the complex structure of mycorrhizal networks by using mathematical modeling and graph theory. To understand the organization and function of mycorrhizal networks, we propose community analysis to identify cohesive subgroups of fungal species that are tightly connected within the network. Furthermore, centrality measures such as degree, betweenness, eccentricity and proximity centrality will be used to identify the most important nodes in the network and to gain insights into how the network changes over time. Understanding the complex structure of mycorrhizal networks is critical for improving our knowledge of plant-fungal interactions and may have important implications for managing ecosystems in the future.}
%These propagation processes are also referred to as social contagion, as they are reminiscent of how a disease is transmitted between individuals. A FAZER}

% Keywords
\keyword{Random network; spread dynamics; Erd\"os-R\'enyi networks; linear threshold model; spectral theory. MUDAR} 

%%%%%%%%%%%%%%%%%%%%%%%%%%%%%%%%%%%%%%%%%%%
\begin{document}

%%%%%%%%%%%%%%%%%%%%%%%%%%%%%%%%%%%%%%%%%%

%%%%%%%%%%%%%%%%%%%%%%%%%%%%%%%%%%%%%%%%%%%%%%%%%%%%%%
%
% Section 1
%
%%%%%%%%%%%%%%%%%%%%%%%%%%%%%%%%%%%%%%%%%%%%%%%%%%%%%%%

\section{Introduction and motivation (800 words)} \label{sec:1}
This section provides an overview of the research topic, establishes the context, and explains the motivation behind the study. It should introduce the research question or problem, discuss its significance, and convey the importance of addressing this issue within the field.
\subsection{Background (300 words)} Relevant literature; historical context; previous studies; current state of the field; identification of knowledge gaps.

\subsection{Research question (200 words)} Formulation of research question or problem; specific objectives; scope and focus of the study.

\subsection{Significance of the study (300 words)} Contribution to the field; potential practical applications; implications for theory development; benefits for stakeholders.

%NOTA: Falta juntar mais detalhe e referencias biblio

%%%%%%%%%%%%%%%%%%%%%%%%%%%%%%%%%%%%%%%%%%%%%%%%%%%%%%
%
% Section 2
%
%%%%%%%%%%%%%%%%%%%%%%%%%%%%%%%%%%%%%%%%%%%%%%%%%%%%%%%
\section{Preliminary Results (600 words)} \label{sec:2}
In this section, present any initial findings or observations from your research that have informed the development of your study. Discuss the implications of these preliminary results and how they have shaped the direction of your research.

\subsection{Initial findings (300 words)} Early observations; data analysis; patterns and trends; unexpected results; hypotheses refinement.

\subsection{Implications of preliminary results (300 words)} Influence on research direction; impact on theoretical framework; informing research questions and objectives; adjustments to methodology.


%%%%%%%%%%%%%%%%%%%%%%%%%%%%%%%%%%%%%%%%%%%%%%%%%%%%%%
%
% Section 3
%
%%%%%%%%%%%%%%%%%%%%%%%%%%%%%%%%%%%%%%%%%%%%%%%%%%%%%%%

\section{Theoretical Procedure (1200 words)}\label{sec:3}
This section presents the development and validation of a theoretical framework or model for the study. This section delves into the underlying assumptions, key concepts, and relationships between them, providing readers with a comprehensive understanding of the rationale behind the proposed approach and its applicability.

\subsection{Conceptual foundations (400 words)}
Key concepts and theories related to community detection in mycorrhizal networks; existing models and frameworks in the literature; limitations or gaps in current understanding.

\subsection{Model development (400 words)}
Proposed theoretical model or framework for community detection in mycorrhizal networks; main components of the model and their relationships; underlying assumptions and rationale for choices.

\subsection{Model validation and application (400 words)}
Validation methods for the proposed model (e.g., simulations, real-world data, comparisons with existing models); application of the model to selected mycorrhizal networks and evaluation of its performance; expected outcomes and potential implications for the field of community detection.



%%%%%%%%%%%%%%%%%%%%%%%%%%%%%%%%%%%%%%%%%%%%%%%%%%%%%%
%
% Section 4
%
%%%%%%%%%%%%%%%%%%%%%%%%%%%%%%%%%%%%%%%%%%%%%%%%%%%%%%%

\section{Methodology (1200 words)}\label{sec:4} 
This section describes the overall research design and methods used to conduct the study, including data collection and analysis techniques. It should provide enough detail for readers to understand your approach and, if applicable, replicate your study.

\subsection{Research design (400 words)} Research approach; quantitative or qualitative methods; sampling strategy; identification of variables; selection of appropriate methods; ethical considerations.

\subsection{Data collection (400 words)} Data sources; primary or secondary data; surveys, interviews, or observations; data collection instruments; data reliability and validity; data collection process.

\subsection{Data analysis (400 words)} Data preprocessing; statistical or thematic analysis; analytical techniques or software; interpretation of findings; addressing biases or limitations in the data.

%%%%%%%%%%%%%%%%%%%%%%%%%%%%%%%%%%%%%%%%%%%%%%%%%%%%%%
%
% Section 5
%
%%%%%%%%%%%%%%%%%%%%%%%%%%%%%%%%%%%%%%%%%%%%%%%%%%%%%%%

\section{Numerical Studies (800 words)} \label{sec:5}
This section focuses on the numerical simulations or experiments conducted as part of your research. Describe the setup of these simulations, present their results, and discuss the validation of your numerical model in relation to real-world data or theoretical predictions.

\subsection{Simulation setup (300 words)} Simulation parameters; model input data; boundary conditions; initial conditions; computational resources; software or tools used.

\subsection{Simulation results (300 words)} Generated data; observed trends or patterns; visualizations or plots; notable outcomes; interpretation of results; deviations from expectations.

\subsection{Model validation (200 words)} Comparison with real-world data or theoretical predictions; error estimation; model limitations or assumptions; sensitivity analysis; validation criteria.

%%%%%%%%%%%%%%%%%%%%%%%%%%%%%%%%%%%%%%%%%%%%%%%%%%%%%%
%
% Section 6
%
%%%%%%%%%%%%%%%%%%%%%%%%%%%%%%%%%%%%%%%%%%%%%%%%%%%%%%%

\section{Results (1000 words)} \label{sec:6}
In this section, present the primary outcomes of your research, including any patterns, trends, or relationships observed in the data. Discuss the results of any statistical tests performed and compare your findings with relevant previous research.
\subsection{Main findings (400 words)} Key outcomes; identified patterns and trends; relationships between variables; support or challenge hypotheses; observed phenomena.

\subsection{Statistical analysis (300 words)} Statistical tests performed; test assumptions; results interpretation; significance levels; effect sizes; confidence intervals.

\subsection{Comparisons with previous research (300 words)} Similarities and differences with existing literature; explanations for discrepancies; contextualization within the field; contribution to knowledge.


%%%%%%%%%%%%%%%%%%%%%%%%%%%%%%%%%%%%%%%%%%
\section{Discussion (800 words)}\label{discussion}
The discussion section provides a detailed analysis of the results, including the interpretation of the findings, their implications, and the connections to the existing literature. It should also address any unexpected outcomes and possible explanations for these observations. This section allows for a deeper understanding of the results in the context of the broader research field and encourages critical thinking about the study's impact and potential future developments.
\subsection{Interpretation of findings (300 words)} Explanation of the main findings; relation between findings and the research question; identification of trends or patterns; assessment of the study's hypotheses.

\subsection{Implications of results (300 words)}Impact of findings on the field; practical applications of the results; theoretical contributions; policy or management implications; relevance to the existing literature.

\subsection{Comparison with existing literature (200 words)}  Comparison of the study's results with previous research findings; similarities and differences with the literature; contextualization of the results within the broader field.

%%%%%%%%%%%%%%%%%%%%%%%%%%%%%%%%%%%%%%%%%%

\section{Conclusion (600 words)}\label{conclusion} 
The conclusion section summarizes the main findings of your study, highlights its contributions to the field, and reiterates the significance of your research. It should also address any limitations and suggest avenues for future research. This section serves to wrap up the paper, emphasizing the key takeaways and providing a clear, concise overview of the work's relevance and implications for the broader research community.

\subsection{Summary of findings (200 words)} Recap of the main findings; restatement of the research question or problem; synthesis of the results and their significance.

\subsection{Contributions to the field (200 words)} Novelty of the research; advancements in theory, methodology, or knowledge; potential impact on the field; gaps filled in the literature.

\subsection{Limitations and future research (200 words)} Acknowledgment of the study's limitations; methodological or data constraints; areas for improvement; suggestions for future research directions or unanswered questions.


%%%%%%%%%%%%%%%%%%%%%%%%%%%%%%%%%%%%%%%%%%
\vspace{6pt} 

%%%%%%%%%%%%%%%%%%%%%%%%%%%%%%%%%%%%%%%%%%
%% optional
%\supplementary{The following supporting information can be downloaded at:  \linksupplementary{s1}, Figure S1: title; Table S1: title; Video S1: title.}

% Only for the journal Methods and Protocols:
% If you wish to submit a video article, please do so with any other supplementary material.
% \supplementary{The following supporting information can be downloaded at: \linksupplementary{s1}, Figure S1: title; Table S1: title; Video S1: title. A supporting video article is available at doi: link.}

%%%%%%%%%%%%%%%%%%%%%%%%%%%%%%%%%%%%%%%%%%
\authorcontributions{For research articles with several authors, a short paragraph specifying their individual contributions must be provided. The following statements should be used ``Conceptualization, X.X. and Y.Y.; methodology, X.X.; software, X.X.; validation, X.X., Y.Y. and Z.Z.; formal analysis, X.X.; investigation, X.X.; resources, X.X.; data curation, X.X.; writing---original draft preparation, X.X.; writing---review and editing, X.X.; visualization, X.X.; supervision, X.X.; project administration, X.X.; funding acquisition, Y.Y. All authors have read and agreed to the published version of the manuscript.'', please turn to the  \href{http://img.mdpi.org/data/contributor-role-instruction.pdf}{CRediT taxonomy} for the term explanation. Authorship must be limited to those who have contributed substantially to the work~reported.}

\funding{Please add: ``This research received no external funding'' or ``This research was funded by NAME OF FUNDER grant number XXX.'' and  and ``The APC was funded by XXX''. Check carefully that the details given are accurate and use the standard spelling of funding agency names at \url{https://search.crossref.org/funding}, any errors may affect your future funding.}

\institutionalreview{In this section, you should add the Institutional Review Board Statement and approval number, if relevant to your study. You might choose to exclude this statement if the study did not require ethical approval. Please note that the Editorial Office might ask you for further information. Please add “The study was conducted in accordance with the Declaration of Helsinki, and approved by the Institutional Review Board (or Ethics Committee) of NAME OF INSTITUTE (protocol code XXX and date of approval).” for studies involving humans. OR “The animal study protocol was approved by the Institutional Review Board (or Ethics Committee) of NAME OF INSTITUTE (protocol code XXX and date of approval).” for studies involving animals. OR “Ethical review and approval were waived for this study due to REASON (please provide a detailed justification).” OR “Not applicable” for studies not involving humans or animals.}

\informedconsent{Any research article describing a study involving humans should contain this statement. Please add ``Informed consent was obtained from all subjects involved in the study.'' OR ``Patient consent was waived due to REASON (please provide a detailed justification).'' OR ``Not applicable'' for studies not involving humans. You might also choose to exclude this statement if the study did not involve humans.

Written informed consent for publication must be obtained from participating patients who can be identified (including by the patients themselves). Please state ``Written informed consent has been obtained from the patient(s) to publish this paper'' if applicable.}

\dataavailability{We encourage all authors of articles published in MDPI journals to share their research data. In this section, please provide details regarding where data supporting reported results can be found, including links to publicly archived datasets analyzed or generated during the study. Where no new data were created, or where data is unavailable due to privacy or ethical re-strictions, a statement is still required. Suggested Data Availability Statements are available in section “MDPI Research Data Policies” at \url{https://www.mdpi.com/ethics}.} 

\acknowledgments{In this section you can acknowledge any support given which is not covered by the author contribution or funding sections. This may include administrative and technical support, or donations in kind (e.g., materials used for experiments).}

\conflictsofinterest{Declare conflicts of interest or state ``The authors declare no conflict of interest.'' Authors must identify and declare any personal circumstances or interest that may be perceived as inappropriately influencing the representation or interpretation of reported research results. Any role of the funders in the design of the study; in the collection, analyses or interpretation of data; in the writing of the manuscript; or in the decision to publish the results must be declared in this section. If there is no role, please state ``The funders had no role in the design of the study; in the collection, analyses, or interpretation of data; in the writing of the manuscript; or in the decision to publish the~results''.} 

%%%%%%%%%%%%%%%%%%%%%%%%%%%%%%%%%%%%%%%%%%
%% Optional
\sampleavailability{Samples of the compounds ... are available from the authors.}

%% Only for journal Encyclopedia
%\entrylink{The Link to this entry published on the encyclopedia platform.}

\abbreviations{Abbreviations}{
The following abbreviations are used in this manuscript:\\

\noindent 
\begin{tabular}{@{}ll}
MDPI & Multidisciplinary Digital Publishing Institute\\
DOAJ & Directory of open access journals\\
TLA & Three letter acronym\\
LD & Linear dichroism
\end{tabular}
}

%%%%%%%%%%%%%%%%%%%%%%%%%%%%%%%%%%%%%%%%%%
%% Optional
\appendixtitles{no} % Leave argument "no" if all appendix headings stay EMPTY (then no dot is printed after "Appendix A"). If the appendix sections contain a heading then change the argument to "yes".
\appendixstart
\appendix
\section[\appendixname~\thesection]{}
\subsection[\appendixname~\thesubsection]{}
The appendix is an optional section that can contain details and data supplemental to the main text---for example, explanations of experimental details that would disrupt the flow of the main text but nonetheless remain crucial to understanding and reproducing the research shown; figures of replicates for experiments of which representative data are shown in the main text can be added here if brief, or as Supplementary Data. Mathematical proofs of results not central to the paper can be added as an appendix.

\begin{table}[H] 
\caption{This is a table caption.\label{tab5}}
\newcolumntype{C}{>{\centering\arraybackslash}X}
\begin{tabularx}{\textwidth}{CCC}
\toprule
\textbf{Title 1}	& \textbf{Title 2}	& \textbf{Title 3}\\
\midrule
Entry 1		& Data			& Data\\
Entry 2		& Data			& Data\\
\bottomrule
\end{tabularx}
\end{table}

\section[\appendixname~\thesection]{}
All appendix sections must be cited in the main text. In the appendices, Figures, Tables, etc. should be labeled, starting with ``A''---e.g., Figure A1, Figure A2, etc.

%%%%%%%%%%%%%%%%%%%%%%%%%%%%%%%%%%%%%%%%%%
\begin{adjustwidth}{-\extralength}{0cm}
%\printendnotes[custom] % Un-comment to print a list of endnotes

\reftitle{References}

% Please provide either the correct journal abbreviation (e.g. according to the “List of Title Word Abbreviations” http://www.issn.org/services/online-services/access-to-the-ltwa/) or the full name of the journal.
% Citations and References in Supplementary files are permitted provided that they also appear in the reference list here. 

%=====================================
% References, variant A: external bibliography
%=====================================
%\bibliography{your_external_BibTeX_file}

%=====================================
% References, variant B: internal bibliography
%=====================================
\begin{thebibliography}{999}

\bibitem[Talukder(2019)]{IEEE} Talukder, A.; Alam, Md.G.R.; Tran, N.H.; Niyato, D.; Park, G.H.; Hong, C.S. Threshold estimation models for linear threshold-based influential user mining in social networks {\em IEEE Access} {\bf 2019}, {\em 7}, 1--21.

\bibitem[Demetrius(2005)]{PhysA} Demetrius, L.; Manke, T. Robustness and network evolution - an entropic principle {\em Physica A} {\bf 2005}, {\em 346}, 682--696. 

\bibitem[Arnold(1994)]{AP} Arnold, L.; Gundlach V.; Demetrius, L. Evolutionary formalism for products of positive random matrices {\em Ann. Probab.} {\bf 1994}, {\em 4}, 859--901. 

\bibitem[Rocha(2004)]{IJMMS} Rocha, J.L.; Sousa Ramos J. Weighted kneading theory of one-dimensional maps with a hole {\em Int. J. Math. Math. Sci.} {\bf 2004}, {\em 38}, 2019-2038.

\bibitem[Rocha(2006)]{NPSC} Rocha, J.L.; Sousa Ramos J. Computing conditionally invariant measures and escape rates {\em Neural, Parallel Sci. Comput.} {\bf 2006}, {\em 14}, 97-114.

\bibitem[Rocha(2021)]{MCS} Rocha, J.L.; Carvalho S. Information transmission and synchronizability in complete networks of systems with linear dynamics {\em Math. Comput. Simul.} {\bf 2021}, {\em 182}, 340-352.

\bibitem[Rocha(2013)]{CMS} Rocha, J.L.; Caneco, A. Mutual information rate and topological order in networks {\em Chaotic Modeling and Simulation, Int. J. Nonlinear Sci.} {\bf 2013}, {\em 4}, 553--562. 

\bibitem[Rocha(2015)]{AMIS} Rocha, J.L.; Gr\'{a}cio, C.; Fernandes, S.; Caneco, A. Spectral and dynamical invariants in a complete clustered network {\em Appl. Math. Inf. Sci.} {\bf 2015}, {\em 9}, 2367--2376. 	
							
\bibitem[Rocha(2023)]{DNC} Rocha, J.L.; Carvalho S. Complete Dynamical Networks: Synchronization, Information Transmission and Topological Order {\em Discontinuity Nonlinearity Complex.} {\bf 2023}, {\em 12}, 99-109.

\bibitem[Krievelevich(2003)]{CPC} Krievelevich, M.; Sudakov, B. The largest eingenvalue of sparse random graphs {\em Comb. Probab. Comput.} {\bf 2003}, {\em 12}, 61-72.

\bibitem[Feigie(2003)]{RSA} Feige, U.; Ofek, E. Spectral techniques applied to sparse random graphs {\em Random Struct. Algorit.} {\bf 2003}, {\em 27}, 251–275. 

% Reference 1
% \bibitem[Author1(year)]{ref-journal}
%Author~1, T. The title of the cited article. {\em Journal Abbreviation} {\bf 2008}, {\em 10}, 142--149.

% Reference 2
% \bibitem[Author2(year)]{ref-book1}
% Author~2, L. The title of the cited contribution. In {\em The Book Title}; Editor 1, F., Editor 2, A., Eds.; Publishing House: City, Country, 2007; pp. 32--58.

% Reference 3
%\bibitem[Author3(year)]{ref-book2}
%Author 1, A.; Author 2, B. \textit{Book Title}, 3rd ed.; Publisher: Publisher Location, Country, 2008; pp. 154--196.

% Reference 4
% \bibitem[Author4(year)]{ref-unpublish}
% Author 1, A.B.; Author 2, C. Title of Unpublished Work. \textit{Abbreviated Journal Name} year, \textit{phrase indicating stage of publication (submitted; accepted; in press)}.

% Reference 5
% \bibitem[Author5(year)]{ref-communication}
% Author 1, A.B. (University, City, State, Country); Author 2, C. (Institute, City, State, Country). Personal communication, 2012.

% Reference 6
%\bibitem[Author6(year)]{ref-proceeding}
% Author 1, A.B.; Author 2, C.D.; Author 3, E.F. Title of presentation. In Proceedings of the Name of the Conference, Location of Conference, Country, Date of Conference (Day Month Year); Abstract Number (optional), Pagination (optional).

% Reference 7
%\bibitem[Author7(year)]{ref-thesis}
%Author 1, A.B. Title of Thesis. Level of Thesis, Degree-Granting University, Location of University, Date of Completion.

% Reference 8
% \bibitem[Author8(year)]{ref-url}
% Title of Site. Available online: URL (accessed on Day Month Year).

\end{thebibliography}

% If authors have biography, please use the format below
%\section*{Short Biography of Authors}
%\bio
%{\raisebox{-0.35cm}{\includegraphics[width=3.5cm,height=5.3cm,clip,keepaspectratio]{Definitions/author1.pdf}}}
%{\textbf{Firstname Lastname} Biography of first author}
%
%\bio
%{\raisebox{-0.35cm}{\includegraphics[width=3.5cm,height=5.3cm,clip,keepaspectratio]{Definitions/author2.jpg}}}
%{\textbf{Firstname Lastname} Biography of second author}

% For the MDPI journals use author-date citation, please follow the formatting guidelines on http://www.mdpi.com/authors/references
% To cite two works by the same author: \citeauthor{ref-journal-1a} (\citeyear{ref-journal-1a}, \citeyear{ref-journal-1b}). This produces: Whittaker (1967, 1975)
% To cite two works by the same author with specific pages: \citeauthor{ref-journal-3a} (\citeyear{ref-journal-3a}, p. 328; \citeyear{ref-journal-3b}, p.475). This produces: Wong (1999, p. 328; 2000, p. 475)

%%%%%%%%%%%%%%%%%%%%%%%%%%%%%%%%%%%%%%%%%%
%% for journal Sci
%\reviewreports{\\
%Reviewer 1 comments and authors’ response\\
%Reviewer 2 comments and authors’ response\\
%Reviewer 3 comments and authors’ response
%}
%%%%%%%%%%%%%%%%%%%%%%%%%%%%%%%%%%%%%%%%%%
\PublishersNote{}
\end{adjustwidth}
\end{document}

